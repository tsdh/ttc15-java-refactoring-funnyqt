\documentclass[submission]{eptcs}
\providecommand{\event}{TTC 2015}

\usepackage[T1]{fontenc}
\usepackage{varioref}
\usepackage{hyperref}

\usepackage{url}
\usepackage{paralist}
\usepackage{graphicx}
\usepackage[cache]{minted}
\newminted{clojure}{fontsize=\fontsize{8}{8},linenos,numbersep=3pt,numberblanklines=false}
\newmintinline{clojure}{fontsize=\small}
\newcommand{\code}{\clojureinline}
\VerbatimFootnotes

\title{Solving the TTC Java Refactoring Case with FunnyQT}
\author{Tassilo Horn
  \institute{Institute for Software Technology, University Koblenz-Landau, Germany}
  \email{horn@uni-koblenz.de}}

\def\titlerunning{Solving the TTC Java Refactoring Case with FunnyQT}
\def\authorrunning{T. Horn}

\begin{document}
\maketitle

\begin{abstract}
  This paper describes the FunnyQT solution to the TTC 2015 Java Refactoring
  transformation case.  The solution solves all core tasks and also the
  extension tasks 1 and 2, and it has been selected as overall winner of this
  case.
\end{abstract}


\section{Introduction}
\label{sec:introduction}

This paper describes the FunnyQT\footnote{\url{http://funnyqt.org}}
~\cite{Horn2013MQWFQ,funnyqt-icgt15} solution of the TTC 2015 Java Refactoring
Case~\cite{java-refactoring-case-desc}.  It solves all core and exception tasks
with the exception of \emph{Extension~3: Detecting Refactoring Conflicts}.  The
solution project is available on
Github\footnote{\url{https://github.com/tsdh/ttc15-java-refactoring-funnyqt}},
and it is set up for easy reproduction on a SHARE image\footnote{The SHARE
  image name is \verb|ArchLinux64_TTC15-FunnyQT_2|}.

FunnyQT is a model querying and transformation library for the functional Lisp
dialect Clojure\footnote{\url{http://clojure.org}}.  Queries and
transformations are Clojure programs using the features provided by the FunnyQT
API.

Clojure provides strong metaprogramming capabilities that are used by FunnyQT
in order to define several \emph{embedded domain-specific languages} (DSL) for
different querying and transformation tasks.

FunnyQT is designed with extensibility in mind.  By default, it supports EMF
models and JGraLab TGraph models.  Support for other modeling frameworks can be
added without having to touch FunnyQT's internals.

The FunnyQT API is structured into several namespaces, each namespace providing
constructs supporting concrete querying and transformation use-cases, e.g.,
model management, pattern matching, out-place transformations, in-place
transformations, bidirectional transformations, and some more.  For solving
this case, FunnyQT's out-place and in-place transformation DSLs have been used.


\section{Solution Description}
\label{sec:solution-description}

The solution consists of three steps:
\begin{compactenum}[1.]
\item Converting the Java code to a program graph
\item Refactoring the program graph
\item Propagating changes in the program graph back to the Java code
\end{compactenum}
All steps are discussed in the following sections.

\subsection{Step 1: Java Code to Program Graph}
\label{sec:step-1:java-to-pg}

The first step in the transformation chain is to create an instance model
conforming to the program graph metamodel predefined in the case description
from the Java source code that should be subject to refactoring.  The FunnyQT
solution does that in two substeps.
\begin{compactenum}[(a)]
\item Parse the Java source code into a model conforming to the EMFText
  JaMoPP\footnote{\url{http://www.jamopp.org/index.php/JaMoPP}} metamodel.
\item Transform the JaMoPP model to a program graph metamodel using a FunnyQT
  out-place transformation.
\end{compactenum}
Step (a) is implemented in the solution namespace
\emph{ttc15-java-refactoring-funnyqt.jamopp}.  It simply sets up JaMoPP and
defines two functions, one for parsing a source tree to a JaMoPP model, and a
second one to synchronize the changes in a JaMoPP model back to the source
tree.  Both just access JaMoPP built-in functionality.

Step (b) is implemented as a FunnyQT out-place transformation.  It creates a
program graph from the JaMoPP model.

The transformation also minimizes the target program graph.  The source JaMoPP
model contains the complete syntax graph of the parsed Java sources including
all their dependencies.  In contrast, the program graph created by the
transformation only contains \textsf{TClass} elements for the Java classes
parsed from source code and direct dependencies used as field type or method
parameter or method return type.  \textsf{TMember} elements are only created
for the methods of directly parsed Java classes, and then only for those
members that are not static because the case description explicitly excludes
statics.  As a result, the program graph contains only the information relevant
to the refactorings and is reasonably small so that it can be visualized which
is nice especially for debugging purposes.

The FunnyQT out-place transformation API used for implementing this task is
quite similar to ATL or QVT Operational Mappings.  There are mapping rules
which receive one or many JaMoPP source elements and create one or many target
program graph elements.

A cutout of the transformation depicting the rules responsible for transforming
fields is given in the following listing.  The transformation receives one
single source model \code|jamopp| and one single target model \code|pg|.

\begin{clojurecode}
(deftransformation jamopp2pg [[jamopp] [pg]]
  ...
  (field2tfielddef
   :from [f 'Field]
   :when (not (static? f))
   :to   [tfd 'TFieldDefinition {:signature (get-tfieldsig f)}])
  (get-tfieldsig
   :from [f 'Field]
   :id   [sig (str (type-name (get-type f)) " " (j/name f))]
   :to   [tfs 'TFieldSignature {:field (get-tfield f)
                                :type  (type2tclass (get-type f))}])
  (get-tfield
   :from [f 'Field]
   :id   [n (j/name f)]
   :to   [tf 'TField {:tName n}]
   (pg/->add-fields! *tg* tf))
  (type2tclass
   :from [t 'Type]
   :disjuncts [class2tclass primitive2tclass])
  ...)
\end{clojurecode}

For each non-static field in the JaMoPP model, the \code|field2tfielddef| rule
creates one \textsf{TFieldDefinition} element in the program graph.  The
signature of this \textsf{TFieldDefinition} is set to the result of calling the
\code|get-tfieldsig| rule.

This rule uses the \code|:id| feature to implement a n:1 semantics.  Only for
each unique string \code|sig| created by concatenating the field's type and
name, a new \textsf{TFieldSignature} is created.  If the rule is called
thereafter for some other field with the same type and name, the existing field
signature created at the first call is returned.  The field signature's
\textsf{field} and \textsf{type} references pointing to a \textsf{TField} and a
\textsf{TClass} respectively are set by calling the \code|get-tfield| and
\code|type2tclass| rules which are not shown because of space limitations.

In total, the transformation consists of 10 rules summing up to 71 lines of
code.  In addition, there are five simple helper functions like \code|static?|,
\code|get-type|, and \code|type-name| that have been used in the above rules
already.

A FunnyQT transformation like the one briefly discussed above returns a map of
traceability information.  This traceability map is used in step~3 of the
overall procedure, i.e., the back-propagation of changes in the program graph
to the Java source code.


\subsection{Step~2: Refactoring of the Program Graph}
\label{sec:step-2:refactoring-pg}

The refactorings are implemented in the solution namespace
\emph{ttc15-java-refactoring-funnyqt.refactor} using FunnyQT in-place
transformation rules which combine patterns to be matched in the model with
actions to be applied to the matched elements.

All rules defined in the following have a parameter \code|pg2jamopp-map-atom|
which is essentially the inverse of the traceability map created by the JaMoPP
to program graph transformation from step~1, i.e., it allows to translate
program graph \textsf{TClass} and \textsf{TMember} elements to the
corresponding JaMoPP \textsf{Class} and \textsf{Member} elements.

\paragraph{Pull Up Member.}

The case description requests \emph{pull-up method} as first refactoring core
task.  However, with respect to the program graph metamodel, there is actually
no difference in pulling up a method (\textsf{TMethodDefinition}) or a field
(\textsf{TFieldDefinition}), i.e., it is possible to define the refactoring
more generally as \emph{pull-up member} (\textsf{TMember}) and have it work for
both fields and methods.  This is what the FunnyQT solution does.

The corresponding \code|pull-up-member| rule is shown in the next listing.  The
rule is overloaded on arity.  There is the version (1) of arity three which
receives the program graph \code|pg|, the inverse lookup map
\code|pg2jamopp-map-atom|, and the JaMoPP resource set \code|jamopp|, and there
is the version (2) of arity four which receives the program graph \code|pg|,
the inverse lookup map atom \code|pg2jamopp-map-atom|, a \textsf{TClass}
\code|super|, and a \textsf{TSignature} \code|sig|.
\begin{clojurecode*}{firstnumber=last}
(defrule pull-up-member
  ([pg pg2jamopp-map-atom jamopp]                                             ;; (1)
   [:extends [(pull-up-member 1)]]                                            ;; pattern
   ((do-pull-up-member! pg pg2jamopp-map-atom super sub member sig others)    ;; action
    jamopp))
  ([pg pg2jamopp-map-atom super sig]                                          ;; (2)
   [super<TClass> -<:childClasses>-> sub -<:signature>-> sig                  ;; pattern
    sub -<:defines>-> member<TMember> -<:signature>-> sig
    :nested [others [super -<:childClasses>-> osub
                     :when (not= sub osub)
                     osub -<:signature>-> sig
                     osub -<:defines>-> omember<TMember> -<:signature>-> sig]]
    :when (seq others)                                                        ;; (a)
    super -!<:signature>-> sig                                                ;; (b)
    :when (= (count (pg/->childClasses super)) (inc (count others)))          ;; (c)
    :when (forall? (partial accessible-from? super)                           ;; (d)
                   (mapcat pg/->access (conj (map :omember others) member)))]
   (do-pull-up-member! pg pg2jamopp-map-atom super sub member sig others)))   ;; action
\end{clojurecode*}

The version (2) is the one which is called by the ARTE test framework whereas
the first version is called when performing the interactive refactoring
extension.

The pattern of the version (2) matches a subclass \code|sub| of class
\code|super| where \code|sub| defines a \code|member| of the given signature
\code|sig|.  A nested pattern is used to match all other subclasses of
\code|super| which also define a member with that signature.  The constraint
(a) ensures that there are in fact other subclasses declaring a member with
signature \code|sig|.  Then the negative application condition (b) defines that
the superclass \code|super| must not define a member of the given \code|sig|
already.  The constraint (c) ensures that all subclasses define a member of the
given \code|sig|, i.e., not only a subset of all subclasses do so.  Lastly, the
constraint (d) makes sure that all field and method definitions accessed by the
member to be pulled up is already accessible from the superclass\footnote{The
  \code|accessible-from?| predicate has been skipped for brevity.}.

The pattern of the arity three variant (1) of the \code|pull-up-member| rule
contains just an \code|:extends| clause specifying that its pattern equals the
pattern defined for the arity four variant.  As said, this variant is used by
the extension task~2 where possible refactorings are to be proposed to the
user.  The difference between the overloaded versions of the
\code|pull-up-member| rule is that version (1) matches \code|super| and
\code|sig| itself whereas these two elements are parameters provided by the
caller (i.e., ARTE) in version (2).

When a match is found, both versions of the rule call the function
\code|do-pull-up-member!| which is defined as follows.

\begin{clojurecode*}{firstnumber=last}
(defn do-pull-up-member! [pg pg2jamopp-map-atom super sub member sig others]
  (doseq [o others]                                                 ;; PG modification
    (doseq [acc (find-accessors pg (:omember o))]
      (pg/->remove-access! acc (:omember o))
      (pg/->add-access! acc member))
    (edelete! (:omember o))
    (pg/->remove-signature! (:osub o) sig))
  (pg/->remove-signature! sub sig)
  (pg/->add-defines! super member)
  (pg/->add-signature! super sig)
  (fn [_]                                                           ;; JaMoPP modification
    (doseq [o others]
      (edelete! (@pg2jamopp-map-atom (:omember o)))
      (swap! pg2jamopp-map-atom dissoc (:omember o)))
    (j/->add-members! (@pg2jamopp-map-atom super) (@pg2jamopp-map-atom member))))

(defn find-accessors [pg tmember]
  (filter #(member? tmember (pg/->access %))
          (pg/all-TMembers pg)))
\end{clojurecode*}

It first applies the changes to the program graph by deleting all duplicate
member definitions from all other subclasses of \code|super| and pulling up the
selected member into \code|super|.  It also updates all accessors of the old
members in order to have them access the single pulled up member.  Lastly, it
returns a closure which performs the equivalent changes in the JaMoPP model and
updates the reference to the inverse lookup map when being called.

A function encapsulating the changes is returned here instead of simply
applying the changes also to the JaMoPP model because the ARTE
\textsf{TestInterface} defines that the back-propagation of changes happens at
a different point in time than the refactoring of the program graph.  Thus, the
solution's \textsf{TestInterface} implementation simply collects the closures
returned by appling the rules in a collection and invokes them in its
\code|synchronizeChanges()| implementation.

Note that the rule's variant (1) immediately invokes the function returned by
\code|do-pull-up-member!|.  This is because this variant is not called by ARTE
but is intended for extension task~2, and with that there is no need to defer
back-propagation.

The other core rule \code|create-superclass| is defined analogously, and the
extension rule \code|extract-superclass| simply combines
\code|create-superclass| with \code|pull-up-member|.

FunnyQT provides built-in functionality to let users steer rule application,
i.e., choose an applicable rule and one of its matches and then apply the rule
to that match.  This feature is used for solving the second extension task of
proposing refactorings to the user.


\subsection{Step~3: Program Graph to Java Code}
\label{sec:step-3:pg-to-java}

The core \code|pull-up-member| and \code|create-superclass| rules both return
closures which perform the refactoring's actions in the JaMoPP model when ARTE
calls the \textsf{TestInterface}'s \code|synchronizeChanges()| method.  Then,
the JaMoPP model needs to be saved to reflect those changes also in the Java
source code files.  This is done by the \code|synchronizeChanges()| method of
the solution's \textsf{TestInterface} implementation.

\begin{minted}[fontsize=\fontsize{8}{8}]{java}
    public boolean synchronizeChanges() {
        try {
            for (IFn synchronizer : synchronizeFns) { synchronizer.invoke(jamoppRS); }
            SAVE_JAVA_RESOURCE_SET.invoke(jamoppRS);
            return true;
        } catch (Exception e) { return false; }
          finally { synchronizeFns.clear(); }
    }
\end{minted}

\code|synchronizedFns| is the list of closures returned by the rules which
simply get invoked and perform the same changes to the JaMoPP model which have
previously been applied to the program graph.  Thereafter, the JaMoPP resource
set is saved which means that the source code files are updated accordingly.


\section{Evaluation \& Conclusion}
\label{sec:evaluation}

In this section, the FunnyQT solution is evaluated according to the criteria
suggested in the case description which was also used as the basis for the open
peer review.

The FunnyQT solution is \emph{correct}, i.e., all tests performed by ARTE pass,
and implements all core tasks.  Thus, it is also \emph{complete} and received a
full score of 60 points.

According to ARTE, the FunnyQT solution runs in less than a tenth of a second
for all test cases on an off-the-shelf laptop so the \emph{performance} seems
to be good.  Nevertheless, the benchmarking performed by the case authors
suggested that all other solutions except for NMF perform even better.
However, all the ARTE test cases are actually too small to provide accurate
numbers.  And in any case, the execution time of the actual refactorings on the
program graph and the back-propagation into the JaMoPP model are completely
negligible when being compared to the time JaMoPP needs to parse the Java
sources, resolve references in the created model, and serialize the model back
to Java again.

Another strong point of the solution is its \emph{conciseness}.  It weights
only 271 NCLOC of FunnyQT/Clojure code for all core and extension tasks and 145
NCLOC of Java code for the \code|TestInterface| implementation class required
by ARTE.

The FunnyQT solution also received a high \emph{extension score} because it
provides runnable implementations for the extensions~1 (\emph{extract
  superclass}) and 2 (\emph{propose refactoring}).

A \emph{main critic} of the solution and FunnyQT in general is that many
developers used languages with C-like syntax such as Java dislike FunnyQT's
Lisp-syntax.  Additionally, its functional emphasis where transformations and
rules are essentially functions which might get composed and passed to
higher-order functions requires a shift from the object-oriented to the
functional paradigm.  Although this provides several benefits it also requires
more learning effort and thus might hinder the adoption of FunnyQT.

Nevertheless, the FunnyQT solution received a reasonably good reviewer score
which paired with its excellent correctness and completeness resulted in making
it the overall winner of this case.


\bibliographystyle{eptcs}
\bibliography{ttc-java-refactoring}
\end{document}

%%% Local Variables:
%%% mode: latex
%%% TeX-master: t
%%% TeX-command-extra-options: "-shell-escape"
%%% End:

%  LocalWords:  parallelizes traceability
